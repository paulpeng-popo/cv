%!TEX program = xelatex
%!TEX root = chinese_cv.tex

\documentclass[11pt, letterpaper]{article}

\hfuzz=0.5pt

% Chinese font settings:
\usepackage{fontspec}
\usepackage{xeCJK}
\setCJKmainfont{Heiti TC}[AutoFakeBold=3,AutoFakeSlant=.4]
\XeTeXlinebreaklocale "zh"
\XeTeXlinebreakskip = 0pt plus 1pt
\defaultCJKfontfeatures{AutoFakeBold=3,AutoFakeSlant=.4}

% Packages:
\usepackage[
    ignoreheadfoot, % set margins without considering header and footer
    top=1.3 cm, % seperaation between body and page edge from the top
    bottom=1.3 cm, % seperation between body and page edge from the bottom
    left=1.4 cm, % seperation between body and page edge from the left
    right=1.4 cm, % seperation between body and page edge from the right
    footskip=1.0 cm, % seperation between body and footer
    % showframe % for debugging 
]{geometry} % for adjusting page geometry
\usepackage[explicit]{titlesec} % for customizing section titles
\usepackage{tabularx} % for making tables with fixed width columns
\usepackage{array} % tabularx requires this
\usepackage[dvipsnames]{xcolor} % for coloring text
\definecolor{primaryColor}{RGB}{0, 79, 144} % define primary color
\usepackage{enumitem} % for customizing lists
\usepackage{fontawesome5} % for using icons
\usepackage{amsmath} % for math
\usepackage[
    pdftitle={Paul Peng's CV},
    pdfauthor={Paul Peng},
    pdfcreator={LaTeX with RenderCV},
    colorlinks=true,
    urlcolor=primaryColor
]{hyperref} % for links, metadata and bookmarks
\usepackage[pscoord]{eso-pic} % for floating text on the page
\usepackage{calc} % for calculating lengths
\usepackage{bookmark} % for bookmarks
\usepackage{lastpage} % for getting the total number of pages
\usepackage{changepage} % for one column entries (adjustwidth environment)
\usepackage{etoolbox}
\usepackage{paracol} % for two and three column entries
\usepackage{ifthen} % for conditional statements
\usepackage{needspace} % for avoiding page brake right after the section title
\usepackage{iftex} % check if engine is pdflatex, xetex or luatex

% Ensure that generate pdf is machine readable/ATS parsable:
\ifPDFTeX
    \input{glyphtounicode}
    \pdfgentounicode=1
    \usepackage[T1]{fontenc}
    \usepackage[utf8]{inputenc}
    \usepackage{lmodern}
\fi

\usepackage[default, type1]{sourcesanspro} 

% Some settings:
\AtBeginEnvironment{adjustwidth}{\partopsep0pt} % remove space before adjustwidth environment
\pagestyle{empty} % no header or footer
\setcounter{secnumdepth}{0} % no section numbering
\setlength{\parindent}{0pt} % no indentation
\setlength{\topskip}{0pt} % no top skip
\setlength{\columnsep}{0.15cm} % set column seperation
\makeatletter
\let\ps@customFooterStyle\ps@plain % Copy the plain style to customFooterStyle
\patchcmd{\ps@customFooterStyle}{\thepage}{
    \color{gray}\textit{\small Paul Peng - Page \thepage{} of \pageref*{LastPage}}
}{}{} % replace number by desired string
\makeatother
\pagestyle{customFooterStyle}

\titleformat{\section}{
    % avoid page braking right after the section title
    \needspace{4\baselineskip}
    % make the font size of the section title large and color it with the primary color
    \Large\color{primaryColor}
}{
}{
}{
    % print bold title, give 0.15 cm space and draw a line of 0.8 pt thickness
    % from the end of the title to the end of the body
    \textbf{#1}\hspace{0.15cm}\titlerule[0.8pt]\hspace{-0.1cm}
}[] % section title formatting

\titlespacing{\section}{
    % left space:
    -1pt
}{
    % top space:
    0.22 cm
}{
    % bottom space:
    0.1 cm
} % section title spacing

% \renewcommand\labelitemi{$\vcenter{\hbox{\small$\bullet$}}$} % custom bullet points
\newenvironment{highlights}{
    \begin{itemize}[
        topsep=0.10 cm,
        parsep=0.10 cm,
        partopsep=0pt,
        itemsep=0pt,
        leftmargin=0.4 cm + 10pt
    ]
}{
    \end{itemize}
} % new environment for highlights

\newenvironment{highlightsforbulletentries}{
    \begin{itemize}[
        topsep=0.10 cm,
        parsep=0.10 cm,
        partopsep=0pt,
        itemsep=0pt,
        leftmargin=10pt
    ]
}{
    \end{itemize}
} % new environment for highlights for bullet entries


\newenvironment{onecolentry}{
    \begin{adjustwidth}{
        0.2 cm + 0.00001 cm
    }{
        0.2 cm + 0.00001 cm
    }
}{
    \end{adjustwidth}
} % new environment for one column entries

\newenvironment{twocolentry}[2][]{
    \onecolentry
    \def\secondColumn{#2}
    \setcolumnwidth{\fill, 4 cm}
    \begin{paracol}{2}
}{
    \switchcolumn \raggedleft \secondColumn
    \end{paracol}
    \endonecolentry
} % new environment for two column entries

\newenvironment{threecolentry}[3][]{
    \onecolentry
    \def\thirdColumn{#3}
    \setcolumnwidth{1.2 cm, \fill, 4 cm}
    \begin{paracol}{3}
    {\raggedright #2} \switchcolumn
}{
    \switchcolumn \raggedleft \thirdColumn
    \end{paracol}
    \endonecolentry
} % new environment for three column entries

\newenvironment{header}{
    \setlength{\topsep}{0pt}\par\kern\topsep\centering\color{primaryColor}\linespread{1.5}
}{
    \par\kern\topsep
} % new environment for the header

\newcommand{\placelastupdatedtext}{% \placetextbox{<horizontal pos>}{<vertical pos>}{<stuff>}
  \AddToShipoutPictureFG*{% Add <stuff> to current page foreground
    \put(
        \LenToUnit{\paperwidth-2 cm-0.2 cm+0.05cm},
        \LenToUnit{\paperheight-1.0 cm}
    ){\vtop{{\null}\makebox[0pt][c]{
        \small\color{gray}\textit{
            Last updated in December 2024
        }\hspace{\widthof{
            Last updated in December 2024
        }}
    }}}%
  }%
}%


% Command of project role style
\newcommand*{\bodyfont}{\fontspec{Heiti TC}}
\definecolor{gray}{HTML}{5D5D5D}
\colorlet{graytext}{gray}
\newcommand*{\projectrole}[1]{{\fontsize{10pt}{1em}\bodyfont\color{graytext} #1}}


% save the original href command in a new command:
\let\hrefWithoutArrow\href

% new command for external links:
\renewcommand{\href}[2]{\hrefWithoutArrow{#1}{\ifthenelse{\equal{#2}{}}{ }{#2 }\raisebox{.15ex}{\footnotesize \faExternalLink*}}}


\begin{document}
\newcommand{\AND}{\unskip
    \cleaders\copy\ANDbox\hskip\wd\ANDbox
    \ignorespaces
}
\newsavebox\ANDbox
\sbox\ANDbox{}

\placelastupdatedtext
\begin{header}
    \fontsize{28 pt}{28 pt}
    \textbf{彭煜博}

    \vspace{0.3 cm}

    \normalsize
    \mbox{{\footnotesize\faMapMarker*}\hspace*{0.13cm}新竹}%
    \kern 0.25 cm%
    \AND%
    \kern 0.25 cm%
    \mbox{\hrefWithoutArrow{mailto:kmes1234@gmail.com}{{\footnotesize\faEnvelope[regular]}\hspace*{0.13cm}kmes1234@gmail.com}}%
    \kern 0.25 cm%
    \AND%
    \kern 0.25 cm%
    \mbox{\hrefWithoutArrow{tel:+886 972-093-238}{{\footnotesize\faPhone*}\hspace*{0.13cm}0972093238}}%
    \kern 0.25 cm%
    \AND%
    \kern 0.25 cm%
    % \mbox{\hrefWithoutArrow{https://popdoom.com/}{{\footnotesize\faLink}\hspace*{0.13cm}popdoom.com}}%
    % \kern 0.25 cm%
    % \AND%
    % \kern 0.25 cm%
    \mbox{\hrefWithoutArrow{https://linkedin.com/in/paulpeng-popo}{{\footnotesize\faLinkedinIn}\hspace*{0.13cm}paulpeng-popo}}%
    \kern 0.25 cm%
    \AND%
    \kern 0.25 cm%
    \mbox{\hrefWithoutArrow{https://github.com/paulpeng-popo}{{\footnotesize\faGithub}\hspace*{0.13cm}paulpeng-popo}}%
\end{header}

\vspace{0.3 cm - 0.3 cm}


% \section{Welcome to RenderCV!}


% \begin{onecolentry}
%     \href{https://rendercv.com}{RenderCV} is a LaTeX-based CV/resume version-control and maintenance app. It allows you to create a high-quality CV or resume as a PDF file from a YAML file, with \textbf{Markdown syntax support} and \textbf{complete control over the LaTeX code}.
% \end{onecolentry}

% \vspace{0.2 cm}

% \begin{onecolentry}
%     The boilerplate content was inspired by \href{https://github.com/dnl-blkv/mcdowell-cv}{Gayle McDowell}.
% \end{onecolentry}


% \section{Quick Guide}


% \begin{onecolentry}
%     \begin{highlightsforbulletentries}

%         \item Each section title is arbitrary and each section contains a list of entries.

%         \item There are 7 unique entry types: \textit{BulletEntry}, \textit{TextEntry}, \textit{EducationEntry}, \textit{ExperienceEntry}, \textit{NormalEntry}, \textit{PublicationEntry}, and \textit{OneLineEntry}.

%         \item Select a section title, pick an entry type, and start writing your section!

%         \item \href{https://docs.rendercv.com/user_guide/}{Here}, you can find a comprehensive user guide for RenderCV.

%     \end{highlightsforbulletentries}
% \end{onecolentry}


\section{教育背景}


\begin{threecolentry}{
        \textbf{碩士}
    }{
        2022.09 – 2024.09
    }
    \textbf{國立成功大學}, 資訊工程研究所
    % \begin{highlights}
    %     \item \textbf{修課內容:} \hspace{0.1cm} 多語暨跨語資訊系統, Linux 核心實作, 計算機結構
    %     \item \textbf{研究:} \hspace{0.1cm} 基於新聞議題自動生成隱含問句之推理問答系統設計與實現
    %     \item \textbf{相關技術:} \hspace{0.1cm} NLP, Knowledge Graph, Neo4j
    % \end{highlights}
\end{threecolentry}


\begin{threecolentry}{
        \textbf{學士}
    }{
        2018.09 – 2022.06
    }
    \textbf{國立中山大學}, 資訊工程學系
    % \begin{highlights}
    %     \item \textbf{修課內容:} \hspace{0.1cm} 數位影像處理, 深度學習, 網路應用程式設計
    %     \item \textbf{專題:} \hspace{0.1cm} 植基於屬性加密之外掛式郵件加密系統
    %     \item \textbf{相關技術:} \hspace{0.1cm} Attribute-Based Encryption, Web Development
    % \end{highlights}
\end{threecolentry}


% \section{Experience}


% \begin{twocolentry}{
%         Cupertino, CA

%         June 2005 – Aug 2007
%     }
%     \textbf{Apple}, Software Engineer
%     \begin{highlights}
%         \item Reduced time to render user buddy lists by 75\% by implementing a prediction algorithm
%         \item Integrated iChat with Spotlight Search by creating a tool to extract metadata from saved chat transcripts and provide metadata to a system-wide search database
%         \item Redesigned chat file format and implemented backward compatibility for search
%     \end{highlights}
% \end{twocolentry}

% \vspace{0.2 cm}

% \begin{twocolentry}{
%         Redmond, WA

%         June 2003 – Aug 2003
%     }
%     \textbf{Microsoft}, Software Engineer Intern
%     \begin{highlights}
%         \item Designed a UI for the VS open file switcher (Ctrl-Tab) and extended it to tool windows
%         \item Created a service to provide gradient across VS and VS add-ins, optimizing its performance via caching
%         \item Built an app to compute the similarity of all methods in a codebase, reducing the time from $\mathcal{O}(n^2)$ to $\mathcal{O}(n \log n)$
%         \item Created a test case generation tool that creates random XML docs from XML Schema
%         \item Automated the extraction and processing of large datasets from legacy systems using SQL and Perl scripts
%     \end{highlights}
% \end{twocolentry}


\section{研究成果}


\begin{samepage}
    \begin{enumerate}[label={[\arabic*]}]
        \item \mbox{\textbf{\textit{Yu-Po Peng}}}, \mbox{Chung-Ping Young}, \mbox{Wen-Hsiang Lu}, "\textbf{Design and Implementation of a Reasoning Question Answering System for Automatically Generating Implicit Questions Based on News Topics}," \textit{Master's Thesis, National Cheng Kung University}, 2024. \hspace*{\fill} \hrefWithoutArrow{http://140.116.245.147:1233/}{\faDesktop} \hrefWithoutArrow{https://hdl.handle.net/11296/w9hyxe}{\faIcon[regular]{file-pdf}}
        \item \mbox{\textbf{\textit{Yu-Po Peng}}}, \mbox{Pin-Chung Chen}, \mbox{Kuan-Yu Ko}, \mbox{Jun-Huei Wang}, \mbox{Shin-Nan Guo}, \mbox{Chun-I Fan}, "\textbf{Secure Extended Mail System Based on Attribute-Based Encryption}," \textit{Cryptology and Information Security Conference}, 2022. \hspace*{\fill} \hrefWithoutArrow{https://github.com/paulpeng-popo/Security_mail/blob/main/CISC2022.pdf}{\faIcon[regular]{file-pdf}}
    \end{enumerate}
\end{samepage}


\section{榮譽}


\begin{threecolentry}{
        2023
    }{
        \hrefWithoutArrow{https://www.datastation.org.tw/cases/65}{\faGlobe}
    }
    \textbf{AI 語音輔助視障者觀展體驗 - 評審肯定獎}, \hspace{0.1 cm} AIGOOD 實戰場域人才選拔競賽
\end{threecolentry}


\begin{threecolentry}{
        2022
    }{
        \hrefWithoutArrow{https://www.ccisa.org.tw/}{\faGlobe}
    }
    \textbf{最佳學生論文獎-佳作}, \hspace{0.1 cm} 第三十二屆全國資訊安全會議
\end{threecolentry}


\begin{threecolentry}{
        2021
    }{
        \hrefWithoutArrow{https://cse.nsysu.edu.tw/p/412-1205-23062.php?Lang=zh-tw}{\faGlobe}
    }
    \textbf{普鴻資訊資安特別獎}, \hspace{0.1 cm} 國立中山大學工學院聯合專題競賽
\end{threecolentry}


\section{開發經歷}


\begin{twocolentry}{
        2024.08 - 2024.09
    }
    \textbf{亞仕丹醫療器材問答助手} \hspace{0.2cm} \hrefWithoutArrow{https://github.com/paulpeng-popo/ragllm}{\faGithub} \hfill \projectrole{主要開發者}
    \begin{highlights}
        \item 在大型語言模型 (LLM) 的基礎上實作 Retrival Argumented Generation (RAG) 的流程,強化模型回答的專業性與準確性,避免誤導使用者。
        \item \textbf{相關技術:} Ollama, Streamlit, LangChain, Chroma
    \end{highlights}
\end{twocolentry}

\vspace{0.2 cm}

\begin{twocolentry}{
        2023.09 - 2024.07
    }
    \textbf{基於新聞議題自動生成隱含問句的推理問答系統設計與實現} \hfill \projectrole{主要開發者}
    \begin{highlights}
        \item 針對新聞議題自動生成隱含問句,並透過知識圖譜進行推理問答,提供使用者更多元的資訊。
        \item \textbf{相關技術:} NLP, Chatbot, Knowledge Graph
    \end{highlights}
\end{twocolentry}

\vspace{0.2 cm}

% \begin{twocolentry}{
%         2023.04 - 2023.06
%     }
%     \textbf{新聞知識圖譜建構系統} \hspace{0.2cm} \hrefWithoutArrow{http://140.116.245.147:8090/}{\faDesktop} \hfill \projectrole{主要維護者}
%     \begin{highlights}
%         \item 接手實驗室的知識圖譜展示系統,追蹤分析各個模組的問題,並進行修正與優化,最後將\textbf{系統建構速度提升 3 倍}。
%         \item 使用工具: Neo4j, HanLP, Flask
%     \end{highlights}
% \end{twocolentry}

% \vspace{0.2 cm}

\begin{twocolentry}{
        2023.02 - 2024.04
    }
    \textbf{J-kumo(TW) 台灣日語學習雲端支援系統} \hspace{0.2cm} \hrefWithoutArrow{https://j-kumo.flld.ncku.edu.tw/}{\faDesktop} \hfill \projectrole{主要維護者}
    \begin{highlights}
        \item 以 Django 整合 6 個舊系統的服務,並利用 shell script 與 docker compose 實現簡易自動化部署流程,\textbf{2023年9月上線至今累計約 70000 次瀏覽使用}。
        \item \textbf{相關技術:} Django, SQL, Nginx, Docker
    \end{highlights}
\end{twocolentry}

\begin{twocolentry}{
        2020.09 - 2021.10
    }
    \textbf{植基於屬性加密之外掛式郵件系統} \hspace{0.2cm} \hrefWithoutArrow{https://github.com/paulpeng-popo/Security_mail}{\faGithub} \hfill \projectrole{主要開發者}
    \begin{highlights}
        \item 在原有的 Gmail 系統上,加上屬性加密流程,既防止郵件內容遭到竊取,又能在已加密的文字上做搜尋。
        \item \textbf{相關技術:} Attribute-Based Encryption, Gmail API, Web Development
    \end{highlights}
\end{twocolentry}


% \section{Technologies}


% \begin{onecolentry}
%     \textbf{Languages:} C++, C, Java, Objective-C, C\#, SQL, JavaScript
% \end{onecolentry}

% \vspace{0.2 cm}

% \begin{onecolentry}
%     \textbf{Technologies:} .NET, Microsoft SQL Server, XCode, Interface Builder
% \end{onecolentry}


\section{課程實作}


\begin{twocolentry}{
        2023 春
    }
    \textbf{Linux 核心專題實作 - fibdrv} \hspace{0.2cm} \hrefWithoutArrow{https://hackmd.io/@sysprog/Hk2NJG3H2}{\faGlobe} \hrefWithoutArrow{https://github.com/paulpeng-popo/fibdrv}{\faGithub}
    \begin{highlights}
        \item 實作費氏數列的 Linux 核心模組,使用 API 分析效能,並嘗試改進運算速度
        \item \textbf{相關技術:} C/C++, Linux Kernel
    \end{highlights}
\end{twocolentry}

\vspace{0.2 cm}

\begin{twocolentry}{
        2022 春
    }
    \textbf{線上聊天服務} \hspace{0.2cm} \hrefWithoutArrow{https://github.com/paulpeng-popo/online_chatting}{\faGithub}
    \begin{highlights}
        \item 使用 Socket 與 Multi-process 實作簡易聊天室,支援多人同時聊天
        \item \textbf{相關技術:} C/C++, Socket Programming
    \end{highlights}
\end{twocolentry}


\end{document}
